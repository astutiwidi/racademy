\documentclass[]{article}
\usepackage{lmodern}
\usepackage{amssymb,amsmath}
\usepackage{ifxetex,ifluatex}
\usepackage{fixltx2e} % provides \textsubscript
\ifnum 0\ifxetex 1\fi\ifluatex 1\fi=0 % if pdftex
  \usepackage[T1]{fontenc}
  \usepackage[utf8]{inputenc}
\else % if luatex or xelatex
  \ifxetex
    \usepackage{mathspec}
  \else
    \usepackage{fontspec}
  \fi
  \defaultfontfeatures{Ligatures=TeX,Scale=MatchLowercase}
\fi
% use upquote if available, for straight quotes in verbatim environments
\IfFileExists{upquote.sty}{\usepackage{upquote}}{}
% use microtype if available
\IfFileExists{microtype.sty}{%
\usepackage{microtype}
\UseMicrotypeSet[protrusion]{basicmath} % disable protrusion for tt fonts
}{}
\usepackage[margin=1in]{geometry}
\usepackage{hyperref}
\hypersetup{unicode=true,
            pdftitle={kmeans.r},
            pdfauthor={Students},
            pdfborder={0 0 0},
            breaklinks=true}
\urlstyle{same}  % don't use monospace font for urls
\usepackage{color}
\usepackage{fancyvrb}
\newcommand{\VerbBar}{|}
\newcommand{\VERB}{\Verb[commandchars=\\\{\}]}
\DefineVerbatimEnvironment{Highlighting}{Verbatim}{commandchars=\\\{\}}
% Add ',fontsize=\small' for more characters per line
\usepackage{framed}
\definecolor{shadecolor}{RGB}{248,248,248}
\newenvironment{Shaded}{\begin{snugshade}}{\end{snugshade}}
\newcommand{\AlertTok}[1]{\textcolor[rgb]{0.94,0.16,0.16}{#1}}
\newcommand{\AnnotationTok}[1]{\textcolor[rgb]{0.56,0.35,0.01}{\textbf{\textit{#1}}}}
\newcommand{\AttributeTok}[1]{\textcolor[rgb]{0.77,0.63,0.00}{#1}}
\newcommand{\BaseNTok}[1]{\textcolor[rgb]{0.00,0.00,0.81}{#1}}
\newcommand{\BuiltInTok}[1]{#1}
\newcommand{\CharTok}[1]{\textcolor[rgb]{0.31,0.60,0.02}{#1}}
\newcommand{\CommentTok}[1]{\textcolor[rgb]{0.56,0.35,0.01}{\textit{#1}}}
\newcommand{\CommentVarTok}[1]{\textcolor[rgb]{0.56,0.35,0.01}{\textbf{\textit{#1}}}}
\newcommand{\ConstantTok}[1]{\textcolor[rgb]{0.00,0.00,0.00}{#1}}
\newcommand{\ControlFlowTok}[1]{\textcolor[rgb]{0.13,0.29,0.53}{\textbf{#1}}}
\newcommand{\DataTypeTok}[1]{\textcolor[rgb]{0.13,0.29,0.53}{#1}}
\newcommand{\DecValTok}[1]{\textcolor[rgb]{0.00,0.00,0.81}{#1}}
\newcommand{\DocumentationTok}[1]{\textcolor[rgb]{0.56,0.35,0.01}{\textbf{\textit{#1}}}}
\newcommand{\ErrorTok}[1]{\textcolor[rgb]{0.64,0.00,0.00}{\textbf{#1}}}
\newcommand{\ExtensionTok}[1]{#1}
\newcommand{\FloatTok}[1]{\textcolor[rgb]{0.00,0.00,0.81}{#1}}
\newcommand{\FunctionTok}[1]{\textcolor[rgb]{0.00,0.00,0.00}{#1}}
\newcommand{\ImportTok}[1]{#1}
\newcommand{\InformationTok}[1]{\textcolor[rgb]{0.56,0.35,0.01}{\textbf{\textit{#1}}}}
\newcommand{\KeywordTok}[1]{\textcolor[rgb]{0.13,0.29,0.53}{\textbf{#1}}}
\newcommand{\NormalTok}[1]{#1}
\newcommand{\OperatorTok}[1]{\textcolor[rgb]{0.81,0.36,0.00}{\textbf{#1}}}
\newcommand{\OtherTok}[1]{\textcolor[rgb]{0.56,0.35,0.01}{#1}}
\newcommand{\PreprocessorTok}[1]{\textcolor[rgb]{0.56,0.35,0.01}{\textit{#1}}}
\newcommand{\RegionMarkerTok}[1]{#1}
\newcommand{\SpecialCharTok}[1]{\textcolor[rgb]{0.00,0.00,0.00}{#1}}
\newcommand{\SpecialStringTok}[1]{\textcolor[rgb]{0.31,0.60,0.02}{#1}}
\newcommand{\StringTok}[1]{\textcolor[rgb]{0.31,0.60,0.02}{#1}}
\newcommand{\VariableTok}[1]{\textcolor[rgb]{0.00,0.00,0.00}{#1}}
\newcommand{\VerbatimStringTok}[1]{\textcolor[rgb]{0.31,0.60,0.02}{#1}}
\newcommand{\WarningTok}[1]{\textcolor[rgb]{0.56,0.35,0.01}{\textbf{\textit{#1}}}}
\usepackage{graphicx,grffile}
\makeatletter
\def\maxwidth{\ifdim\Gin@nat@width>\linewidth\linewidth\else\Gin@nat@width\fi}
\def\maxheight{\ifdim\Gin@nat@height>\textheight\textheight\else\Gin@nat@height\fi}
\makeatother
% Scale images if necessary, so that they will not overflow the page
% margins by default, and it is still possible to overwrite the defaults
% using explicit options in \includegraphics[width, height, ...]{}
\setkeys{Gin}{width=\maxwidth,height=\maxheight,keepaspectratio}
\IfFileExists{parskip.sty}{%
\usepackage{parskip}
}{% else
\setlength{\parindent}{0pt}
\setlength{\parskip}{6pt plus 2pt minus 1pt}
}
\setlength{\emergencystretch}{3em}  % prevent overfull lines
\providecommand{\tightlist}{%
  \setlength{\itemsep}{0pt}\setlength{\parskip}{0pt}}
\setcounter{secnumdepth}{0}
% Redefines (sub)paragraphs to behave more like sections
\ifx\paragraph\undefined\else
\let\oldparagraph\paragraph
\renewcommand{\paragraph}[1]{\oldparagraph{#1}\mbox{}}
\fi
\ifx\subparagraph\undefined\else
\let\oldsubparagraph\subparagraph
\renewcommand{\subparagraph}[1]{\oldsubparagraph{#1}\mbox{}}
\fi

%%% Use protect on footnotes to avoid problems with footnotes in titles
\let\rmarkdownfootnote\footnote%
\def\footnote{\protect\rmarkdownfootnote}

%%% Change title format to be more compact
\usepackage{titling}

% Create subtitle command for use in maketitle
\providecommand{\subtitle}[1]{
  \posttitle{
    \begin{center}\large#1\end{center}
    }
}

\setlength{\droptitle}{-2em}

  \title{kmeans.r}
    \pretitle{\vspace{\droptitle}\centering\huge}
  \posttitle{\par}
    \author{Students}
    \preauthor{\centering\large\emph}
  \postauthor{\par}
      \predate{\centering\large\emph}
  \postdate{\par}
    \date{2019-05-22}


\begin{document}
\maketitle

\begin{Shaded}
\begin{Highlighting}[]
\CommentTok{#import library}
\KeywordTok{library}\NormalTok{(cluster)}
\KeywordTok{library}\NormalTok{(gridExtra)}
\end{Highlighting}
\end{Shaded}

\begin{verbatim}
## Warning: package 'gridExtra' was built under R version 3.5.3
\end{verbatim}

\begin{Shaded}
\begin{Highlighting}[]
\KeywordTok{library}\NormalTok{(factoextra)}
\end{Highlighting}
\end{Shaded}

\begin{verbatim}
## Warning: package 'factoextra' was built under R version 3.5.3
\end{verbatim}

\begin{verbatim}
## Loading required package: ggplot2
\end{verbatim}

\begin{verbatim}
## Warning: package 'ggplot2' was built under R version 3.5.3
\end{verbatim}

\begin{verbatim}
## Welcome! Related Books: `Practical Guide To Cluster Analysis in R` at https://goo.gl/13EFCZ
\end{verbatim}

\begin{Shaded}
\begin{Highlighting}[]
\KeywordTok{library}\NormalTok{(tidyverse)}
\end{Highlighting}
\end{Shaded}

\begin{verbatim}
## Warning: package 'tidyverse' was built under R version 3.5.3
\end{verbatim}

\begin{verbatim}
## -- Attaching packages ----------------------------------------------------------- tidyverse 1.2.1 --
\end{verbatim}

\begin{verbatim}
## v tibble  2.1.1     v purrr   0.3.2
## v tidyr   0.8.3     v dplyr   0.8.1
## v readr   1.3.1     v stringr 1.4.0
## v tibble  2.1.1     v forcats 0.4.0
\end{verbatim}

\begin{verbatim}
## Warning: package 'tibble' was built under R version 3.5.3
\end{verbatim}

\begin{verbatim}
## Warning: package 'tidyr' was built under R version 3.5.3
\end{verbatim}

\begin{verbatim}
## Warning: package 'readr' was built under R version 3.5.3
\end{verbatim}

\begin{verbatim}
## Warning: package 'purrr' was built under R version 3.5.3
\end{verbatim}

\begin{verbatim}
## Warning: package 'dplyr' was built under R version 3.5.3
\end{verbatim}

\begin{verbatim}
## Warning: package 'stringr' was built under R version 3.5.3
\end{verbatim}

\begin{verbatim}
## Warning: package 'forcats' was built under R version 3.5.3
\end{verbatim}

\begin{verbatim}
## -- Conflicts -------------------------------------------------------------- tidyverse_conflicts() --
## x dplyr::combine() masks gridExtra::combine()
## x dplyr::filter()  masks stats::filter()
## x dplyr::lag()     masks stats::lag()
\end{verbatim}

\begin{Shaded}
\begin{Highlighting}[]
\CommentTok{#import data}
\KeywordTok{data}\NormalTok{(agriculture)}
\NormalTok{df<-agriculture}

\CommentTok{#data apa yang kita miliki}
\KeywordTok{dim}\NormalTok{(df)}
\end{Highlighting}
\end{Shaded}

\begin{verbatim}
## [1] 12  2
\end{verbatim}

\begin{Shaded}
\begin{Highlighting}[]
\KeywordTok{head}\NormalTok{(df,}\DecValTok{10}\NormalTok{)}
\end{Highlighting}
\end{Shaded}

\begin{verbatim}
##        x    y
## B   16.8  2.7
## DK  21.3  5.7
## D   18.7  3.5
## GR   5.9 22.2
## E   11.4 10.9
## F   17.8  6.0
## IRL 10.9 14.0
## I   16.6  8.5
## L   21.0  3.5
## NL  16.4  4.3
\end{verbatim}

\begin{Shaded}
\begin{Highlighting}[]
\KeywordTok{ggplot}\NormalTok{(}\DataTypeTok{data =}\NormalTok{ df, }\KeywordTok{aes}\NormalTok{(}\DataTypeTok{x =}\NormalTok{ x, }\DataTypeTok{y =}\NormalTok{ x)) }\OperatorTok{+}
\StringTok{  }\KeywordTok{geom_point}\NormalTok{()}
\end{Highlighting}
\end{Shaded}

\includegraphics{kmeans_files/figure-latex/unnamed-chunk-1-1.pdf}

\begin{Shaded}
\begin{Highlighting}[]
\CommentTok{#apakah ada data yang hilang}
\KeywordTok{sum}\NormalTok{(}\KeywordTok{is.na}\NormalTok{(df))}
\end{Highlighting}
\end{Shaded}

\begin{verbatim}
## [1] 0
\end{verbatim}

\begin{Shaded}
\begin{Highlighting}[]
\CommentTok{#apakah data perlu dinormalisasi}
\NormalTok{dfnorm <-}\StringTok{ }\KeywordTok{scale}\NormalTok{(df)}

\CommentTok{#apakah data bisa digunakan untuk melakukan clustering}
\NormalTok{distance <-}\StringTok{ }\KeywordTok{get_dist}\NormalTok{(dfnorm, }\DataTypeTok{method =} \StringTok{"euclidean"}\NormalTok{ )}
\KeywordTok{fviz_dist}\NormalTok{(distance, }\DataTypeTok{gradient =} \KeywordTok{list}\NormalTok{(}\DataTypeTok{low =} \StringTok{"#00AFBB"}\NormalTok{, }\DataTypeTok{mid =} \StringTok{"white"}\NormalTok{, }\DataTypeTok{high =} \StringTok{"#FC4E07"}\NormalTok{))}
\end{Highlighting}
\end{Shaded}

\includegraphics{kmeans_files/figure-latex/unnamed-chunk-1-2.pdf}

\begin{Shaded}
\begin{Highlighting}[]
\CommentTok{#proses clustering}
\NormalTok{k2 <-}\StringTok{ }\KeywordTok{kmeans}\NormalTok{(dfnorm, }\DataTypeTok{centers =} \DecValTok{2}\NormalTok{, }\DataTypeTok{nstart =} \DecValTok{25}\NormalTok{)}
\NormalTok{k3 <-}\StringTok{ }\KeywordTok{kmeans}\NormalTok{(dfnorm, }\DataTypeTok{centers =} \DecValTok{3}\NormalTok{, }\DataTypeTok{nstart =} \DecValTok{25}\NormalTok{)}
\NormalTok{k4 <-}\StringTok{ }\KeywordTok{kmeans}\NormalTok{(dfnorm, }\DataTypeTok{centers =} \DecValTok{4}\NormalTok{, }\DataTypeTok{nstart =} \DecValTok{25}\NormalTok{)}
\NormalTok{k5 <-}\StringTok{ }\KeywordTok{kmeans}\NormalTok{(dfnorm, }\DataTypeTok{centers =} \DecValTok{5}\NormalTok{, }\DataTypeTok{nstart =} \DecValTok{25}\NormalTok{)}
\NormalTok{k6 <-}\StringTok{ }\KeywordTok{kmeans}\NormalTok{(dfnorm, }\DataTypeTok{centers =} \DecValTok{6}\NormalTok{, }\DataTypeTok{nstart =} \DecValTok{25}\NormalTok{)}


\NormalTok{p1 <-}\StringTok{ }\KeywordTok{fviz_cluster}\NormalTok{(k2, }\DataTypeTok{geom =} \StringTok{"point"}\NormalTok{, }\DataTypeTok{data =}\NormalTok{ dfnorm) }\OperatorTok{+}\StringTok{ }\KeywordTok{ggtitle}\NormalTok{(}\StringTok{"k = 2"}\NormalTok{)}
\NormalTok{p2 <-}\StringTok{ }\KeywordTok{fviz_cluster}\NormalTok{(k3, }\DataTypeTok{geom =} \StringTok{"point"}\NormalTok{, }\DataTypeTok{data =}\NormalTok{ dfnorm) }\OperatorTok{+}\StringTok{ }\KeywordTok{ggtitle}\NormalTok{(}\StringTok{"k = 3"}\NormalTok{)}
\NormalTok{p3 <-}\StringTok{ }\KeywordTok{fviz_cluster}\NormalTok{(k4, }\DataTypeTok{geom =} \StringTok{"point"}\NormalTok{, }\DataTypeTok{data =}\NormalTok{ dfnorm) }\OperatorTok{+}\StringTok{ }\KeywordTok{ggtitle}\NormalTok{(}\StringTok{"k = 4"}\NormalTok{)}
\NormalTok{p4 <-}\StringTok{ }\KeywordTok{fviz_cluster}\NormalTok{(k5, }\DataTypeTok{geom =} \StringTok{"point"}\NormalTok{, }\DataTypeTok{data =}\NormalTok{ dfnorm) }\OperatorTok{+}\StringTok{ }\KeywordTok{ggtitle}\NormalTok{(}\StringTok{"k = 5"}\NormalTok{)}
\NormalTok{p5 <-}\StringTok{ }\KeywordTok{fviz_cluster}\NormalTok{(k6, }\DataTypeTok{geom =} \StringTok{"point"}\NormalTok{, }\DataTypeTok{data =}\NormalTok{ dfnorm) }\OperatorTok{+}\StringTok{ }\KeywordTok{ggtitle}\NormalTok{(}\StringTok{"k = 6"}\NormalTok{)}
\KeywordTok{grid.arrange}\NormalTok{(p1, p2, p3, p4, p5, }\DataTypeTok{nrow =} \DecValTok{2}\NormalTok{)}
\end{Highlighting}
\end{Shaded}

\includegraphics{kmeans_files/figure-latex/unnamed-chunk-1-3.pdf}

\begin{Shaded}
\begin{Highlighting}[]
\CommentTok{#mencari jumlah kluster yang optimal menggunakan Elbow}
\KeywordTok{set.seed}\NormalTok{(}\DecValTok{123}\NormalTok{)}
\NormalTok{wss <-}\StringTok{ }\ControlFlowTok{function}\NormalTok{(k) \{}
  \KeywordTok{kmeans}\NormalTok{(dfnorm, k, }\DataTypeTok{nstart =} \DecValTok{10}\NormalTok{ )}\OperatorTok{$}\NormalTok{tot.withinss}
\NormalTok{\}}
\NormalTok{k.values <-}\StringTok{ }\DecValTok{1}\OperatorTok{:}\DecValTok{11}
\NormalTok{wss_values <-}\StringTok{ }\KeywordTok{map_dbl}\NormalTok{(k.values, wss)}
\KeywordTok{plot}\NormalTok{(k.values, wss_values,}
     \DataTypeTok{type=}\StringTok{"b"}\NormalTok{, }\DataTypeTok{pch =} \DecValTok{19}\NormalTok{, }\DataTypeTok{frame =} \OtherTok{FALSE}\NormalTok{, }
     \DataTypeTok{xlab=}\StringTok{"Number of clusters K"}\NormalTok{,}
     \DataTypeTok{ylab=}\StringTok{"Total within-clusters sum of squares"}\NormalTok{)}
\end{Highlighting}
\end{Shaded}

\includegraphics{kmeans_files/figure-latex/unnamed-chunk-1-4.pdf}

\begin{Shaded}
\begin{Highlighting}[]
\CommentTok{#mencari jumlah kluster yang optimal menggunakan Silhouette}
\NormalTok{avg_sil <-}\StringTok{ }\ControlFlowTok{function}\NormalTok{(k) \{}
\NormalTok{  km.res <-}\StringTok{ }\KeywordTok{kmeans}\NormalTok{(dfnorm, }\DataTypeTok{centers =}\NormalTok{ k, }\DataTypeTok{nstart =} \DecValTok{25}\NormalTok{)}
\NormalTok{  ss <-}\StringTok{ }\KeywordTok{silhouette}\NormalTok{(km.res}\OperatorTok{$}\NormalTok{cluster, }\KeywordTok{dist}\NormalTok{(df))}
  \KeywordTok{mean}\NormalTok{(ss[, }\DecValTok{3}\NormalTok{])}
\NormalTok{\}}
\NormalTok{k.values <-}\StringTok{ }\DecValTok{2}\OperatorTok{:}\DecValTok{11}
\NormalTok{avg_sil_values <-}\StringTok{ }\KeywordTok{map_dbl}\NormalTok{(k.values, avg_sil)}

\KeywordTok{plot}\NormalTok{(k.values, avg_sil_values,}
     \DataTypeTok{type =} \StringTok{"b"}\NormalTok{, }\DataTypeTok{pch =} \DecValTok{19}\NormalTok{, }\DataTypeTok{frame =} \OtherTok{FALSE}\NormalTok{, }
     \DataTypeTok{xlab =} \StringTok{"Number of clusters K"}\NormalTok{,}
     \DataTypeTok{ylab =} \StringTok{"Average Silhouettes"}\NormalTok{)}
\end{Highlighting}
\end{Shaded}

\includegraphics{kmeans_files/figure-latex/unnamed-chunk-1-5.pdf}

\begin{Shaded}
\begin{Highlighting}[]
\KeywordTok{fviz_nbclust}\NormalTok{(dfnorm, kmeans, }\DataTypeTok{method =} \StringTok{"silhouette"}\NormalTok{)}
\end{Highlighting}
\end{Shaded}

\includegraphics{kmeans_files/figure-latex/unnamed-chunk-1-6.pdf}


\end{document}
